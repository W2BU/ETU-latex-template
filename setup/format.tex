% Установка языков документа
\setdefaultlanguage[spelling=modern]{russian}
\setotherlanguage{english}

% Установка TNR для различных стилей текста
\defaultfontfeatures{Mapping=tex-text}
\setmonofont{Times New Roman}
\setromanfont{Times New Roman}
\setmainfont{Times New Roman}
\newfontfamily\cyrillicfont{Times New Roman}

\urlstyle{same}

% Полуторный интервал
\onehalfspacing

% Красная строка
\setlength{\parindent}{1.25cm}

% Подсветка ссылок
\hypersetup{
    colorlinks=false,
    urlcolor=blue,
    filecolor=blue,
    citecolor=green,
    urlcolor=blue
}

% Папка с изображениями
\graphicspath{{./img/}}
% Длина изображений по умолчанию
\setkeys{Gin}{width=16cm}

% Форматирование подписей к картинками, листингам, таблицам
\captionsetup[figure]{name={Рисунок}, labelsep = endash, font = small, justification=centering}
\captionsetup[table]{name={Таблица}, labelsep = endash, font = small, justification=raggedleft, singlelinecheck=false, position = top}
\captionsetup[listing]{name={Листинг}, labelsep = colon, font = small, justification=raggedright, singlelinecheck=false, position = top}

% Форматирование заголовков глав и подглав
\titleformat
	{\section}
	[hang]
	{\normalfont\bfseries\center}
	{\arabic{section}. }
	{1ex}
	{\MakeUppercase}
\titlespacing
	{\section}
	{\parindent}
	{5ex}
	{4ex}
\titleformat
	{\subsection}
	[hang]
	{\normalfont\bfseries}
	{\arabic{section}.\arabic{subsection}. }
	{1ex}
	{}
\titlespacing
	{\subsection}
	{\parindent}
	{4ex}
	{4ex}

% Нумерация рисунков и таблиц включает номер главы
\counterwithin{figure}{section}
\counterwithin{table}{section}
% Можно включить и для формул
%\counterwithin{equation}{section}

% Установка шрифта для листингов minted
\setmonofont{FiraCode-Regular.ttf}[
	SizeFeatures={Size=10},
	Path = setup/,
	Contextuals=Alternate
]

% Настройка подсветки кода 
\usemintedstyle{gruvbox-light} % делает подсветку для кода

% Убираем жирный шрифт из названия ToC (Table of content)
\renewcommand*\cfttoctitlefont{\normalfont\bfseries\MakeUppercase}
% Убираем жирный шрифт из записей в ToC
\renewcommand{\cftsecfont}{}
% Убираем жирный шрифт из номеров страниц в ToC
\renewcommand{\cftsecpagefont}{}

% Уменьшаем расстояние между элементами списка 
\setlist[enumerate]{itemsep=1.0pt}

% Минимизируем количество переносов
\tolerance = 500
\hyphenpenalty = 20000
\emergencystretch = 1cm
% Вставка изображения \image{}{}[][]

% Обязательные аргументы:
% 	1. Название изображения в папке img без расширения
% 	2. Подпись к рисунку

% Опциональные аргументы:
% 	1. Размер картинки. По умолчанию: 16cm
% 	2. label для ссылок
\NewDocumentCommand\image{mmoo}{
    \begin{figure}[H]
        \centering
        \includegraphics[width=#3, keepaspectratio]{#1}
        \caption{#2}
        \label{#4}
    \end{figure}
}

% Заголовок для лабораторной работы \sectionlab{}
% Отличается от стандартного  измененным стилем и отсутствием нумерации
%
% Обязательные аргументы:
% 	1. Текст заголовка
\NewDocumentCommand\sectionlab{m}{
    \titleformat
	{\section}
	[hang]
	{\normalfont\bfseries\filouter}
	{}
	{0pt}
	{}
\titlespacing
	{\section}
	{\parindent}
	{5ex}
	{4ex}

    \section{#1}

    \titleformat
	{\section}
	[hang]
	{\normalfont\bfseries\center}
	{\arabic{section}. }
	{1ex}
	{\MakeUppercase}
\titlespacing
	{\section}
	{\parindent}
	{5ex}
	{4ex}
}

% Вставка заголовка приложения \sectionappendix[]
% Автоматически нумерует приложения по латинским буквам. Позволяет получать полное название приложения с гиперссылкой через \nameref

% Опциональные аргументы:
% 	1. label для ссылок 
\makeatletter
\NewDocumentCommand\sectionappendix{o}{
    \newpage
    \refstepcounter{section}
    \stepcounter{appendix_counter}
    \section*{Приложение \Alph{appendix_counter}}
    \addcontentsline{toc}{section}{\protect\numberline{}Приложение \Alph{appendix_counter}}
    \edef\@currentlabelname{Приложение \Alph{appendix_counter}}
    \phantomsection
    \label{#1}
}
\makeatother

% Макрос для ссылки на Рисунок в форме "Рисунок 1" \picref{}

% Обязательный аргументы:
% 	1. label картинки, на которую ссылка
\NewDocumentCommand{\picref}{m}{
    \hyperref[{#1}]{Рисунок~\ref*{#1}}
}

% Макрос для ссылки на Листинг в форме "Листинг 1" \lstref{}

% Обязательный аргументы:
% 	1. label картинки, на которую ссылка
\NewDocumentCommand{\lstref}{m}{
    \hyperref[{#1}]{Листинг~\ref*{#1}}
}

% Макрос для ссылки на Таблицу в форме "Таблица 1" \tblref{}

% Обязательный аргументы:
% 	1. label картинки, на которую ссылка
\NewDocumentCommand{\tblref}{m}{
    \hyperref[{#1}]{Таблица~\ref*{#1}}
}


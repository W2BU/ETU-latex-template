% Поддержка Русского языка
\usepackage{polyglossia}

% Для 14го размера текста
\usepackage{extsizes}

% Для полуторного интервала
\usepackage{setspace}

% Для вставки pdf файлов
\usepackage{pdfpages}

% Настройка подписей
\usepackage{caption}

% Математические символы и специальные environments для лемм и прочего
\usepackage{amsmath}
\usepackage{amssymb}
\usepackage{mathtools}

% Ссылки и гиперссылки
\usepackage[unicode]{hyperref}

% Отступ слева 1.25см
\usepackage{indentfirst}

% Для настройки списков
\usepackage{enumitem}

% Вставка картинок
\usepackage{graphicx}
% Для опции "H" картинки, чтобы она вставала там же, где указана в сыром тексте
\usepackage{float}

% Форматирование заголовков
\usepackage{titlesec}

% Библиография 
% \usepackage[citestyle=numeric-comp, sorting=none, style=ieee, backend=biber]{biblatex}

% Библиография по ГОСТ 2008
\usepackage[style=gost-numeric, backend=biber, language=auto, autolang=other, sorting=none]{biblatex}

% Таблица с выравниванием по ширине
\usepackage{tabularx}
% Длинные таблицы, которые не помещаются на одну страницу
\usepackage{longtable}

% Разбиение ячеек и столбцов на несколько строк или колонок
\usepackage{multirow}
\usepackage{multicol}

% Оформление листингов
\usepackage[newfloat]{minted}
\usepackage{listings}
\usepackage{lstfiracode}
\usepackage{color}

% Настройка стиля ToC
\usepackage{tocloft}

% Улучшенное форматирование pdf 
\usepackage{microtype}

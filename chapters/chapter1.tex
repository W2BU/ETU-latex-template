\section{Первая глава}

\subsection{Первая подглава первой главы}
Далее будут описаны примеры использования основных команд шаблона.

Вставка картинки (\picref{pic:example1}) и ссылка на нее:

\image{example_img}{Подпись к картинке}[8cm][pic:example1]

Вставка формулы \eqref{eq:example1} и ссылка на нее:

\begin{equation}
    \label{eq:example1}
    \frac{\delta f}{\delta t} = \lim_{h \rightarrow 0} \left[ \frac{f (t + h) - f(t)}{h} \right]
\end{equation}

Вставка таблицы(\tblref{tbl:example1}) и ссылки на нее:

\begin{table}[H]
    \centering
    \caption{Пример таблицы}
    \label{tbl:example1}
    \begin{tabular}{| c | c | c |}
        \hline
        row & row & row \\
        \hline
        row & row & row \\
        \hline
        row & row & row \\
        \hline
        row & row & row \\
        \hline
        row & row & row \\
        \hline
    \end{tabular}
\end{table}

Листинг minted (\lstref{code:example1}) и ссылка на него:

\begin{listing}[H]
    \caption{Пример листинга}
    \label{code:example1}
    \inputminted[breaklines=true, framesep=10pt, fontsize=\footnotesize, firstline=1, lastline=8]{python}{listing/code_sample.py}
\end{listing}

Листинг listings (\lstref{code:example2}) и ссылка на него:

\begin{listing}[H]
    \caption{Пример листинга}
    \label{code:example2}
    \lstinputlisting[language=Python, firstline=1, lastline=8]{listing/code_sample.py}
\end{listing}

Ссылка на приложение (\nameref{appendix:calc})

Пример цитаты \cite{domanovdi}

Пример нескольких цитат \cite{domanovdi,duportail:alu,fsrf40}

\subsection{Вторая подглава первой главы}

\subsection{Третья подглава первой главы}